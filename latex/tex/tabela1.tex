\begin{center}
\setcaptionmargin{1cm}
\scriptsize
\begin{longtable}{lcccc}
\caption{Cronograma das Atividades}\\
\hline \hline \\[-2ex]
\multicolumn{1}{c}{Periodo} &
\multicolumn{1}{c}{Atividade planejada} &
\multicolumn{1}{c}{Status} &


\\[0.5ex] \hline
\\[-1.8ex]

\endfirsthead

\multicolumn{3}{c}{\footnotesize{{\slshape{{\tablename} \thetable{}}} - Continuacao}}\\[0.5ex]

\hline \hline\\[-2ex]

\multicolumn{1}{c}{Coluna1} &
\multicolumn{1}{c}{Coluna2} &
\multicolumn{1}{c}{Coluna3} 

\\[0.5ex] \hline
\\[-1.8ex]

\endhead

\multicolumn{3}{l}{{\footnotesize{Continua na proxima pagina\ldots}}}\\
\endfoot
\hline

\endlastfoot

1-2 meses & Revis\~ao da disserta\c{c}\~ao de mestrado & \checkmark \\

3-9 meses & Estudo das t\'ecnicas de reconhecimento de padr\~oes, morfologia matem\'atica e topologia digital & em andamento\\

5-9 meses & Disciplina: Forma\c{c}\~ao e Evolu\c{c}\~ao de Gal\'axias & \checkmark \\

6-9 meses & Aprendizado e implementa\c{c}\~ao da t\'ecnica de tessela\c{c}\~ao de Voronoi & \\

6-9 meses & Pr\'e-processamento e caracteriza\c{c}\~ao dos dados de uma amostra & \\

9-12 meses & Estudar para a qualifica\c{c}\~ao &  \checkmark \\

7-10 meses & Implementa\c{c}\~ao das t\'ecnicas de reconhecimento de padr\~oes, morfologia matem\'atica e topologia digital & \\

12-16 meses & Aplica\c{c}\~ao dos algoritmos completos a amostra de gal\'axias do SDSS & \\

17-20 meses & An\'alise dos resultado & \\

18-22 meses & Adapta\c{c}\~ao do algoritmo para trabalhar com dados espectrofotom\'etricos & \\

18-22 meses & Sele\c{c}\~ao de uma amostra espectrofotom\'etrica & \\

22-26 meses & Aplica\c{c}\~ao dos algoritmos a amostra espectrofotom\'etrica & \\

26-30 meses & An\'alise dos Resultados & \\

30-36 meses & Reda\c{c}\~ao e defesa da tese & \\

\label{tabela1}
\end{longtable}
\end{center}